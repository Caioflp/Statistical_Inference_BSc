\documentclass[a4paper,12pt, notitlepage]{report}
\usepackage[utf8]{inputenc}
\usepackage{natbib}
\usepackage{amssymb}
\usepackage{amsmath}
\usepackage{enumitem}
\usepackage[portuguese]{babel}
\usepackage{textcomp}

%%%%%%%%%%%%%%%%%%%% Notation stuff
\newcommand{\indep}{\perp \!\!\! \perp} %% indepence
\newcommand{\pr}{\operatorname{Pr}} %% probability
\newcommand{\vr}{\operatorname{Var}} %% variance
\newcommand{\rs}{X_1, X_2, \ldots, X_n} %%  random sample
\newcommand{\irs}{X_1, X_2, \ldots} %% infinite random sample
\newcommand{\rsd}{x_1, x_2, \ldots, x_n} %%  random sample, realised
\newcommand{\Sm}{\bar{X}_n} %%  sample mean, random variable
\newcommand{\sm}{\bar{x}_n} %%  sample mean, realised
\newcommand{\Sv}{\bar{S}^2_n} %%  sample variance, random variable
\newcommand{\sv}{\bar{s}^2_n} %%  sample variance, realised
\newcommand{\bX}{\boldsymbol{X}} %%  random sample, contracted form (bold)
\newcommand{\bx}{\boldsymbol{x}} %%  random sample, realised, contracted form (bold)
\newcommand{\bT}{\boldsymbol{T}} %%  Statistic, vector form (bold)
\newcommand{\bt}{\boldsymbol{t}} %%  Statistic, realised, vector form (bold)
\newcommand{\emv}{\hat{\theta}_{\text{EMV}}}

% Title Page
\title{Exercícios de Revisão: A2.}
\author{Disciplina: Inferência Estatística \\ Professor: Luiz Max de Carvalho}

\begin{document}
\maketitle

\section{PO-KE-MON.}
Suponha que a Liga Internacional de Pokemon (LIP) tenha um sistema de \textit{pokescores} que podem assumir qualquer valor real. 
Quanto maior o \textit{pokescore} de uma jogadora, mais alto no ranking mundial ela está.
A liga se organiza em times de $n$ jogadores.

Para entrar na liga, um time precisa ter um \textit{pokescore} médio superior a $\theta_0$.
Suponha que os \textit{pokescores} são distribuídos de acordo com uma distribuição Normal com média $\theta$ e variância $\sigma^2$, conhecida.
Queremos desenvolver um método para incluir times num torneio automaticamente, baseado nos \textit{pokescores} dos seus integrantes.

\begin{itemize}
 \item[(a)] Encontre uma quantidade pivotal para $\theta$;
 \item[(b)] Utilizando a quantidade do item anterior, construa um intervalo de confiança de $95\%$ para $\theta$;
 \item[(c)] A partir do intervalo encontrado, é possível testar $H_0: \theta \leq \theta_0$? Como?
 \item[(d)] Se $\sigma^2$ fosse desconhecida, como você modificaria o teste do item anterior?
 \item[(e)] Se aplicarmos os testes em (c) e (d) para selecionar times automaticamente, seremos injustos com alguns times, isto é, vamos deixar de incluir times que de fato se encaixam na condição de seleção.
 Com que probabilidade isso acontece?
 \item[(f)] Se quisermos diminuir a probabilidade do item anterior, o que podemos fazer? Que consequências isso tem?
 \end{itemize}


\section{Acertando a agenda.}

Astolfo quer saber quanto tempo leva para produzir uma lâmpada em sua fábrica.
Bruna, sua assistente, decide medir o tempo de fabricação (em horas) de $n$ lâmpadas aleatórias $\rs$ e acredita que a duração do processo  segue uma distribuição exponencial com parâmetro $\theta>0$, i.e., $f(x \mid \theta) = \theta\exp(-\theta x)$ para $x>0$.
Suponha que essas medidas formam uma amostra aleatória.

Cada máquina só consegue produzir uma lâmpada de cada vez, e trabalha de forma ininterrupta.
Bruna decide verificar se consegue fabricar 48 lâmpadas por dia em cada máquina.

\begin{itemize}
 \item[(a)] Escreva a hipótese alternativa e nula para este teste como função do parâmetro da exponencial e mostre que o poder do teste de hipóteses que rejeita $H_0$ se $S_n = \sum_{i=1}^nX_i \ge c$ é uma função decrescente de $\theta$
 \item[(b)] Encontre um valor de $c$ que faça este teste ter um tamanho $\alpha_0$ pré-definido.
 \item[(c)] Mostre como encontrar o número mínimo de medidas para que o poder do teste em 40 minutos seja ao menos 90\%.
 \item[(d)] Encontre uma estatística pivotal para $\theta$;
 \item[(e)] Use a estatística encontrada no item anterior para construir um intervalo de confiança  de $\gamma$\%.
\end{itemize}


\section{Questões sobre modelos lineares.}
\begin{itemize}
 \item[(a)] Suponha que medimos a potência $X$, em cavalos de potência,  e o consumo em $Km/L$, $Y$, dos motores de $n$ veículos da última edição da Revista Quadro Rodas.
 Proponha um teste para estudar se o coeficiente angular entre as duas variáveis é positivo;
 \item[(b)] Se a potência do motor de um fusquinha é $x_0$ cavalos de potência, mostre como obter um intervalo de confiança para o consumo médio;
 \item[(c)] Compare o intervalo do item anterior com o intervalo de predição para o consumo do motor do fusquinha, $y_0$;
 \item[(d)] O modelo $E[Y] = \beta_0 + \beta_1X + \beta_2X^2 + \beta_3X^3$ é linear?
 Como você ajustaria este modelo a $n$ pares de dados $(x_i, y_i)$?
 \item[(e)]$^\ast$ Considere uma regressão linear simples.
 Descreva uma estratégia de transformação da variável independente, $X$, de modo que $\hat{\beta_0}$ e $\hat{\beta_1}$ sejam independentes. 
\end{itemize}

\section{Pamonha é coisa séria.}

Ainda no processo de validação de seu selo de Pamonha Gourmet\textregistered, Palmirinha agora tem outra preocupação: o equipamento que mede a concentração de amido na pamonha tem um erro de medição não desprezível.
O manual do medidor de concentração de amido diz que o erro de medição é Normal com média $0$ e variância $1$.
\begin{itemize}
 \item[(a)] Proponha um experimento para determinar se o medidor está corretamente calibrado;
 \item[(b)] A partir do experimento no item anterior, proponha um teste para verificar se a variância do erro de medição é diferente do especificado no manual.
 Enuncie sua hipótese nula, alternativa e estatística de teste claramente;
  \item[(c)] Valciclei obteve dados e realizou o teste proposto. 
 Obteve um p-valor de $0.005$. 
 Escreveu para Palmirinha: ``A probabilidade de o medidor estar descalibrado é de 99.5\%''.
 Ele acertou em sua conclusão? Justifique;
 \item[(d)] Suponha que o outro assistente de Palmirinha, Adryelson, repita o mesmo experimento utilizando outro medidor.
 Da mesma forma que antes, proponha um teste para comparar as variâncias dos medidores de Valciclei e Adryelson.
\end{itemize}
 
\end{document}          
