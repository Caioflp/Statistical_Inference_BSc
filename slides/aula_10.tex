\section{Viés}
\begin{frame}{Viés}
 Em Estatística, a palavra viés tem um significado preciso e tem a ver com a esperança da distribuição de um estimador.
 \begin{defn}[Estimador não-viesado]
  \label{def:biased_estimator}
  Um estimador $\delta(\bX)$ de uma função $g(\theta)$ é dito~\textbf{não-viesado} se $E_{\theta}[\delta(\bX)] = g(\theta)$ para todo $\theta \in \Omega$.
  Um estimador que não atende a essa condição é dito~\textit{viesado}.
  O~\textbf{viés} de $\delta$ é definido como $B_\delta(\theta) := E_{\theta}[\delta(\bX)] - g(\theta)$.
 \end{defn}
 \begin{exemplo}[Tempos de falha de lâmpadas]
 Lembremos do exemplo das lâmpadas da fábrica de Astolfo. 
 Neste caso, não é difícil mostrar que $E[\hat{\theta}_{\text{EMV}}] = \frac{n}{n-1} \theta = 3\theta/2$.
 Desta forma, o viés do EMV é $B_{\hat{\theta}_{\text{EMV}}}(\theta) = 3\theta/2 -\theta = \theta/2$.
 É possível encontrar $\delta(\bX)$ não-viesado? Esse estimador é bom?
 \end{exemplo}

\end{frame}

\begin{frame}{Estimadores não-viesados sempre?}
Quando avaliamos estimadores, o erro quadrático médio e o viés são alguns~\textit{aspectos} a serem considerados, mas há um compromisso (\textit{trade-off}) entre eles, de certa forma.
 \begin{obs}[Erro quadrático, variância e viés]
  \label{rmk:bias_variance_mse}
  \begin{equation*}
   R(\theta, \delta) = \vr_\theta(\delta) + \left[B_\delta(\theta)\right]^2.
  \end{equation*}
 \end{obs}
 
 No exemplo das lâmpadas, é possível mostrar que $\delta_2(\bX) = 1/S$ tem o menor EQM, mas tem viés $B_{\delta_2}(\theta) = \frac{n-2}{n-1}\theta = \theta/2$, assim como o EMV.

\end{frame}

\begin{frame}{Estimador não-viesado da variância}
A variância amostral como a temos definido até aqui é viesada. 
Uma pequena modificação leva a um estimador não viesado da variância.
\begin{theo}[Estimador não-viesado da variância]
 Seja $\bX = \{ \rs \}$ uma amostra aleatória, com $E[X_1] = m$ e $\vr(X_1) = v < \infty$.
 Então
 \begin{equation*}
  \delta_1(\bX) = \frac{1}{n-1} \sum_{i=1}^n \left(X_i - \bar{X}_n \right)^2
 \end{equation*}
é um estimador não-viesado de $v$.
\end{theo}
\textbf{Prova:} usar a igualdade
$$ \sum_{i=1}^n \left(X_i - m \right)^2 = \sum_{i=1}^n \left(X_i - \bar{X}_n \right)^2 + n\left(\bar{X}_n - m \right)^2$$
e usar a linearidade da esperança e o fato de que temos uma amostra aleatória.
\end{frame}

\begin{frame}{Nem tudo são flores}

Não-viesamento é uma característica desejável, mas nem sempre um estimador não-viesado (i) existe ou (ii) é um bom estimador.

\begin{itemize}
 \item Não existência.
 Exemplo: $\rs \sim \operatorname{Bernoulli}(p)$, estimador para $\sqrt{p}$?
 
 \item Estimador não-viesado ruim: $X\sim \operatorname{Geometrica}(p)$.
 Quais as propriedades do estimador não viesado, $\delta(X)$?
\end{itemize}
\end{frame}


\section{Eficiência}

\begin{frame}{Escolhendo entre desenhos amostrais}
\begin{exemplo}[Estudando chegada de clientes]
 Exemplo 8.8.1 em DeGroot.
 Suponha que Palmirinha esteja interessada em estudar quantos clientes chegam à sua loja de pamonha num  determinado intervalo.
 Para isso, ela vai modelar o fenômeno como um processo de Poisson:
 $$ Y(\Delta_t) \sim\operatorname{Poisson}(\theta\Delta_t),$$
 isto é, o número $Y$ de clientes num intervalo de tempo $\Delta_t$ tem distribuição Poisson com média $\theta\Delta_t$.
 Palmirinha pode
 \begin{itemize}
  \item Fixar um número $n$ de clientes a serem observados e marcar o tempo, $X$ que leva para chegarem $n$ clientes ou;
  \item Fixar um determinado intervalo de tempo, $t$, e contar o número $Y$ de clientes que chegam neste intervalo.
 \end{itemize}
\textbf{Pergunta:} qual desenho é melhor para estimar $\theta$?
\end{exemplo}
\end{frame}


\begin{frame}{Informação de Fisher}
Como medir a quantidade de informação (sobre um parâmetro $\theta$) contida em uma amostra aleatória?
A (matriz de) informação de Fisher oferece a resposta.
\begin{defn}
 \label{def:Fisher_information}
 Seja $X$ uma variável aleatória com f.d.p/f.m.p. $f(x\mid\theta)$, $\theta \in \Omega \subseteq \mathbb{R}$.
 Suponha que $f(x\mid\theta)$ é duas vezes diferenciável com respeito a $\theta$.
 Defina $\lambda(x\mid \theta) = \log f(x\mid\theta)$ e 
 \begin{equation}
  \lambda^\prime(x\mid \theta) = \frac{\partial \lambda(x\mid \theta)}{\partial \theta}\quad\text{e}\quad \lambda^{\prime\prime}(x\mid \theta) = \frac{\partial^2 \lambda(x\mid \theta)}{\partial \theta^2}.
 \end{equation}
Definimos a~\textbf{informação de Fisher} como
\begin{equation}
 \label{eq:Fisher_information}
 I(\theta) = E_\theta\left[\{\lambda^{\prime\prime}(x\mid \theta)\}^2\right] \stackrel{\text{(1)}}{=} -E_\theta\left[\lambda^{\prime\prime}(x\mid \theta)\right] = \vr_\theta\left(\lambda^{\prime}(x\mid \theta) \right).
\end{equation}
\end{defn}
\textbf{Prova de $\stackrel{\text{(1)}}{=}$}: diferenciar sob o sinal da integral e usar a regra da cadeia.
\end{frame}

\begin{frame}{Informação de Fisher: exemplos}
 \begin{itemize}
  \item Bernoulli;
  \item Normal;
 \end{itemize}

\end{frame}

\begin{frame}{Informação de Fisher: exemplos}
 \begin{itemize}
  \item Bernoulli;
  \begin{equation*}
   I(p) = \frac{1}{p(1-p)}.
  \end{equation*}
  
  \item Normal;
  
  \begin{equation*}
  I(\mu) = \frac{1}{\sigma^2}.
  \end{equation*}
 \end{itemize}

\end{frame}

\begin{frame}{Informação de Fisher de uma amostra aleatória}

\begin{theo}
 \label{thm:Fisher_information_random_sample}
 Seja $\bX = \{ \rs \}$ uma amostra aleatória e seja $I_n(\theta) = E_\theta \left[-\lambda_n^{\prime\prime}(\bX \mid \theta) \right]$ a informação de Fisher da amostra.
 Então
 \begin{equation*}
  I_n(\theta) = nI(\theta).
 \end{equation*} 
\end{theo}
\textbf{Prova:} Usar as propriedades do $\log$, da derivada e a lei de esperanças.
Ver De Groot, Teorema 8.8.2.
\end{frame}

\begin{frame}{Voltando ao dilema de Palmirinha}
Podemos usar a informação de Fisher para analisar os desenhos propostos por Palmirinha.
Não é difícil derivar
\begin{equation*}
 I_X(\theta) = \frac{n}{\theta^2}\quad\text{e}\quad I_Y(\theta) = \frac{t}{\theta}.
\end{equation*}
Portanto, os desenhos são equivalentes se $n = t\theta$, o que não ajuda muito, já que $\theta$ é desconhecido.
Por outro lado, vemos que neste caso não é possível decidir entre os desenhos baseado apenas na informação de Fisher.

\textbf{Extra:} faça uma análise Bayesiana deste problema, derivando a esperança~\textit{a priori} da informação de Fisher sob os dois desenhos.
 \end{frame}

 \begin{frame}{O Teorema de Crámer-Rao}
Outro uso importante da informação de Fisher é encontrar uma cota inferior para a variância de um estimador.
Para isso, empregamos um dos resultados mais importantes da Estatística:
\begin{theo}[Teorema de Crámer-Rao\footnote{Em homenagem ao estatístico indo-estadunidense Calyampudi Radhakrishna Rao (1920-) e ao matemático sueco Harald Cramér (1893--1985).}]
 \label{thm:Cramer_Rao_theorem}
 Seja $\bX = \{ \rs \}$ uma amostra aleatória com f.d.p./f.m.p $f(x\mid\theta)$, com as mesmas premissas da definição~\ref{def:Fisher_information}.
 Suponha que $T = r(\bX)$ é uma estatística com variância finita.
 Seja $m(\theta) = E_\theta(T)$ uma função diferenciável de $\theta$.
 Então,
 \begin{equation}
  \label{eq:Cramer_Rao_theorem}
  \vr_\theta(T) \geq \frac{\left[m^\prime(\theta)\right]^2}{nI(\theta)},
 \end{equation}
 com igualdade apenas se existem $u$ e $v$ tal que 
 \[ T = u(\theta)\lambda_n^\prime(\bX \mid \theta) + v(\theta). \]
\end{theo}
\textbf{Prova:} Usar Cauchy-Schwarz e diferenciar sob o sinal da integral.
\end{frame}

\begin{frame}{Um corolário útil}
Se $T$ é um estimador não-viesado, temos uma expressão útil para a cota de Crámer-Rao.
\begin{obs}[Variância de uma estimador não-viesado]
 Se $T$ é um estimador não-viesado de $\theta$, temos
 \begin{equation*}
  \vr_\theta(T) \geq \frac{1}{nI(\theta)}
 \end{equation*}
\end{obs}
\textbf{Prova:} $T$ é não viesado $\implies$ $m(\theta) = \theta$ $\implies$ $m^\prime(\theta) = 1\: \forall\: \theta \in \Omega \qed$
\end{frame}


\begin{frame}{Eficiência}
Com esse Teorema de Crámer-Rao em mãos, estamos em posição de definir um critério de otimalidade para estimadores.
\begin{defn}[Estimador eficiente]
 \label{def:efficient_estimator}
 Um estimador $\delta(\bX)$ é dito~\textbf{eficiente} de (sua esperança) $m(\theta)$ se 
 \begin{equation*}
    \vr_\theta(\delta) = \frac{\left[m^\prime(\theta)\right]^2}{nI(\theta)}.
 \end{equation*}
\end{defn}

\begin{exemplo}
 Seja $\rs$ uma amostra aleatória de uma distribuição Poisson com parâmetro $\theta$.
 Podemos mostrar que $\bar{X}_n$ é um estimador eficiente de $\theta$.
\end{exemplo}
\end{frame}

\begin{frame}{Distribuição assintótica de um estimador eficiente}
Podemos usar o TCL para estudar a distribuição assintótica de um estimador eficiente.
\begin{theo}[Distribuição assintótica de um estimador eficiente]
 Assumindo as condições de regularidade usuais, considere $\delta$ um estimador eficiente de $m(\theta)$.
 Assuma também que $m^\prime(\theta) \neq 0\: \forall\: \theta \in \Omega$.
 Então a distribuição assintótica de 
 \[\frac{\sqrt{nI(\theta)}}{m^\prime(\theta)} \left[ \delta - m(\theta) \right] \]
 é normal padrão.
 
\end{theo}
\textbf{Prova:} Ver De Groot, Teorema 8.8.4.
Escrever $E_\theta[\delta]$ e $\vr_\theta(\delta)$ explicitamente, usar a condição $\delta = u(\theta)\lambda_n^\prime(\bX \mid \theta) + v(\theta)$, e aplicar as leis de esperanças e variâncias. 

\begin{obs}[Normalidade Assintótica do EMV]
 Supondo que o EMV possa ser derivado ao resolver a equação $\lambda_n^\prime(\bX \mid \theta) = 0$ e que $\lambda_n^{\prime\prime}(\bX \mid \theta)$ e $\lambda_n^{\prime\prime\prime}(\bX \mid \theta)$ satisfazem certas condições técnicas, $\sqrt{nI(\theta)} \left(\hat{\theta}_{\text{EMV}}-\theta\right)^2$ tem distribuição aproximadamente normal padrão.
\end{obs}

\end{frame}




\begin{frame}{O que aprendemos?}
\begin{itemize}
  \item[\faLightbulbO] Viés;
    
    ``Um estimador viesado é aquele cuja esperança não coincide com a função estimada''
    
   \item[\faLightbulbO] Informação de Fisher;
   
   ``A informação de Fisher é uma quantidade derivada de uma distribuição que mede a quantidade de informação contida em uma amostra aleatória advinda desta distribuição''
   
    \item[\faLightbulbO] Crámer-Rao;
    
    ``A desigualdade de Crámer-Rao dá uma cota inferior para a variância de um estimador''    
    
    \item[\faLightbulbO] Distribuição assintótica de estimadores eficientes (e EMV);
    
    ``Sob condições de regularidade, vale um TCL para estimadores eficientes e para o EMV''    
    
   
  \end{itemize}
 \end{frame}

\begin{frame}{Leitura recomendada}
\begin{itemize}
 \item[\faBook] De Groot seções 8.7 e 8.8;
 \item[\faBook] $^\ast$ Casella \& Berger (2002), seção 7.3.
 \item[\faBook] $^\ast$ Schervish (1995), Teorema  5.13.
 \item {\large\textbf{Exercícios recomendados}}
 \begin{itemize}
  \item[\faBookmark] De Groot.
  \begin{itemize}
   \item Seção 8.7: exercícios 4, 6, 11 e 13;
   \item Seção 8.8: exercícios 5, 7 e 10.
  \end{itemize}   
  \end{itemize}
 \end{itemize} 
\end{frame}

