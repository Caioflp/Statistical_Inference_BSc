\section*{Inferência Estatística}
%%%%%%%%%%%%%%%%%%%%%%%%%%%%%%%%%%%
%%%%%%%%%%%%%%%%%%%%%%%%%%%%%%%%%%%
\begin{frame}{O que é e para que serve Inferência Estatística?}

\begin{itemize}
 \item[\faQuestion] Esta moeda é justa?
 \item[\faQuestion] Esta droga ``funciona''?
 \item[\faQuestion] Quantos casos de Dengue teremos mês que vem?
 \item[\faQuestion] Renda básica universal aumenta o PIB?
\end{itemize}

Todas essas perguntas podem ser abordadas com as ferramentas que a Estatística nos fornece.

\begin{ideia}[A gramática da Ciência]
\label{idea:statistics_grammar_science}
\textbf{A Estatística é a gramática da Ciência}\footnote{Título do livro de Karl Pearson (1857--1936) (\href{https://en.wikipedia.org/wiki/The_Grammar_of_Science}{``The Grammar of Science''}), publicado em 1982.}.
O mundo é incerto; medições são imperfeitas.
A Estatística é a linguagem que nos permite expressar e quantificar as incertezas associadas às afirmações científicas através da teoria de probabilidades\footnote{Chamada por E.T. Jaynes (1922-1998) de lógica da Ciência (\href{https://www.cambridge.org/gb/academic/subjects/physics/theoretical-physics-and-mathematical-physics/probability-theory-logic-science}{``Probability Theory: The Logic of Science''}).}.
\end{ideia}
\end{frame}
%%%%%%%%%%%%%%%%%%%%%%%%%%%%%%%%%%%
\begin{frame}{Modelo estatístico: definição informal}
\begin{defn}[Modelo estatístico: informal]
\label{def:statistical_model_informal}
DeGroot, def 7.1.1, pág. 377
Um modelo estatístico consiste na identificação de variáveis aleatórias de interesse (observáveis e potencialmente observáveis), na especificação de uma distribuição conjunta para as variáveis aleatórias observáveis e na identificação dos parâmetros ($\theta$) desta distribuição conjunta.
Às vezes é conveniente assumir que os parâmetros são variáveis aleatórias também, mas para isso é preciso especificar uma distribuição conjunta para $\theta$.
\end{defn}
 
\end{frame}
%%%%%%%%%%%%%%%%%%%%%%%%%%%%%%%%%%%
\begin{frame}{Modelo estatístico: definição formal}
\begin{defn}[Modelo estatístico: formal]
\label{def:statistical_model_formal}
\href{https://projecteuclid.org/download/pdf_1/euclid.aos/1035844977}{McCullagh, 2002}.
Seja $\mathcal{X}$ um espaço amostral qualquer, $\Theta$ um conjunto não-vazio arbitrário e $\mathcal{P}(\mathcal{X})$ o conjunto de todas as distribuições de probabilidade em $\mathcal{X}$.
 Um modelo estatístico~\underline{paramétrico} é uma função $P : \Theta \to \mathcal{P}(\mathcal{X})$, que associa a cada $\theta \in \Theta$ uma distribuição de probabilidade $P_\theta$ em $\mathcal{X}$.
\end{defn}
\textbf{Exemplos}:
\begin{itemize}
 \item Faça $\mathcal{X} = \mathbb{R}$ e $\Theta = (-\infty, \infty)\times (0, \infty)$.
 Dizemos que $P$ é um modelo\footnote{Note o abuso de notação: estritamente falando, $P_\theta$  é uma~\textbf{medida} de probabilidade e não uma~\textit{densidade} como apresentamos aqui.} estatístico normal se para cada $\theta = \{\mu, \sigma^2\} \in \Theta$,
 $$P_{\theta}(x) \equiv \frac{1}{\sqrt{2\pi}\sigma}\exp\left(-\frac{(x-\mu)^2}{2\sigma^2}\right), \: x \in \mathbb{R}.$$
 \item Faça $\mathcal{X} = \mathbb{N}\cup \{0\}$ e $\Theta = (0, \infty)$.
 $P$ é um modelo estatístico Poisson se para $\lambda \in \Theta$,
 $$P_{\lambda}(k) \equiv \frac{e^{-\lambda}\lambda^k}{k!}, \: k = 0, 1, \ldots$$
\end{itemize} 
\end{frame}

%%%%%%%%%%%%%%%%%%%%%%%%%%%%%%%%%%%
\begin{frame}{Exemplo: como sempre, moedas.}
 \begin{pergunta}[Esta moeda é justa?]
 \label{qst:moeda_justa}
  Suponha que uma moeda tenha sido lançada dez vezes, obtendo o seguinte resultado:
  \begin{equation*}
   KKKCKCCCKC
  \end{equation*}
\begin{itemize}
 \item[a)] Esta moeda é justa?
 \item[b)] Quanto eu espero ganhar se apostar R\$ 100,00 que é justa? 
\end{itemize}
 \end{pergunta}
 Podemos formalizar o problema ao, por exemplo, assumir que cada lançamento é uma variável aleatória Bernoulli com probabilidade de cara ($K$), $p$.
 Desta forma $X_i = 1$ se o lançamento deu cara e $X_i = 0$ caso contrário.
 E queremos saber se $p = 1/2$.
 Por ora, não temos as ferramentas necessárias para responder a essa pergunta, mas voltaremos a ela no futuro.
\end{frame}
%%%%%%%%%%%%%%%%%%%%%%%%%%%%%%%%%%%
\begin{frame}{Inferência Estatística}
\begin{defn}[Afirmação probabilística]
\label{def:probabilistic_assertion}
 Dizemos que uma afirmação é probabilística quando ela utiliza conceitos da teoria de probabilidade para falar de um objeto.
 Exemplos: 
 \begin{itemize}
  \item $\pr( \bar{Y}_n \in (0, 1)) \leq 2^{-n}$;
  \item $E[X \mid Y = y] = 2y + 3$;
  \item $\vr(X) = 4p^2$.
  \item $\pr(\vr(X) \leq 4p^2 ) \leq p^2$
 \end{itemize}
\end{defn}
\begin{defn}[Inferência Estatística]
\label{def:statistical_inference}
 Uma inferência estatística é uma~\underline{afirmação probabilística} sobre uma ou mais partes de um modelo estatístico.
 Considerando o exemplo~\ref{qst:moeda_justa}, queremos saber:
 \begin{itemize}
  \item Quantos lançamentos até termos $80\%$ de certeza de que a moeda é justa?
%   \item Se $p$ é a probabilidade de obter cara num dado lançamento e $\hat{p}$ é nossa estimativa para $p$, quanto vale $E[\hat{p}]$? E $\vr(\hat{p})$?
  \item Quanto vale $E[\bar{X}_n]$;
  \item $\pr(X_{n} = 1 \mid X_{n-1} = 1)$. 
 \end{itemize}
\end{defn}
\end{frame}
%%%%%%%%%%%%%%%%%%%%%%%%%%%%%%%%%
\begin{frame}{Estatística}
\begin{defn}[Estatística]
\label{def:statistic}
 Suponha que temos uma coleção de variáveis aleatórias $\rs \in \boldsymbol X \subseteq \mathbb{R}^n$ e uma função $r: \boldsymbol X \to \mathbb{R}^m$.
 Dizemos que a variável aleatória $T = r(\rs)$ é uma~\textbf{estatística}.
\end{defn}
São exemplos de estatísticas:
\begin{itemize}
 \item A média amostral, $\bar{X}_n$;
 \item A soma, $\sum_{i=1}^n X_i$;
 \item O mínimo, $\min(\rs)$;
 \item $r(\rs) = a, \: \forall \rs, \: \, a \in \mathbb{R}$.
\end{itemize}  
\end{frame}
%%%%%%%%%%%%%%%%%%%%%%%%%%%%%%%%%
\begin{frame}{Tipos de Inferência Estatística}
\begin{itemize}
 \item \textbf{Predição}: prever o valor de uma variável aleatória (ainda) não observada; No exemplo~\ref{qst:moeda_justa}, qual será o valor do próximo lançamento, $X_{n+1}$;
 \item \textbf{Decisão Estatística}: Acoplamos o modelo estatístico a uma decisão a ser tomada. Devo emprestar esta moeda ao Duas-Caras? Aqui, temos a~\textit{noção} de~\textbf{risco}.;
 \item \textbf{Desenho experimental}: Quantas vezes é preciso lançar esta moeda para ter 95\% de certeza de que ela é (ou não) justa? Quantas pessoas precisam tomar uma droga para sabermos se ela funciona? Onde devemos cavar para procurar ouro/petróleo?;
\end{itemize}
\end{frame}
%%%%%%%%%%%%%%%%%%%%%%%%%%%%%%%%%
\begin{frame}{O que aprendemos?}
\begin{itemize}
 \item[\faLightbulbO] Modelo estatístico;
 \item[\faLightbulbO] Inferência Estatística;
 \item[\faLightbulbO] Estatística (amostral);
 \item[\faLightbulbO] Tipos de inferências:
 \begin{itemize}
  \item Predição;
  \item Decisão;
  \item Desenho experimental.
 \end{itemize}
%  \item[\faLightbulbO] Estimador:
\end{itemize}
\end{frame}
%%%%%%%%%%%%%%%%%%%%%%%%%%%%%%%%%
\begin{frame}{Leitura recomendada}
\begin{itemize}
 \item[\faBook] DeGroot seção 7.1;
 \item[\faFilePdfO] $^\ast$ \href{https://projecteuclid.org/download/pdf_1/euclid.aos/1035844977}{McCullagh, 2002}.
 \item[\faForward] Próxima aula: DeGroot, seção 7.2;
\end{itemize} 
\end{frame}
%%%%%%%%%%%%%%%%%%%%%%%%%%%%%%%%%
