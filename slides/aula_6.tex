\section*{Estimador de máxima verossimilhança (EMV)}
\begin{frame}{Tópicos da aula}
 \begin{itemize}
  \item Estimador de máxima verossimilhança (EMV);
  \begin{itemize}
     \item Existência e unicidade;
     \item Invariância do EMV;
     \item Consistência do EMV;
  \end{itemize}
  \item Limitações;
 \end{itemize}
\end{frame}

\begin{frame}{Estimador de máxima verossimilhança (EMV)}
 \begin{defn}[Estimador de máxima verossimilhança]
 \label{def:MLE}
  Para cada possível vetor (de observações) $\boldsymbol{x}$, seja $\delta(\boldsymbol{x}) \in \Omega$ um valor de $\theta \in \Omega$ de modo que a função de verossimilhança, $L(\theta) \propto f(\boldsymbol{x} \mid \theta)$, atinge o máximo.
  Dizemos que $\hat{\theta} = \delta(\boldsymbol{X})$ é o~\textbf{estimador de máxima verossimilhança} de $\theta$ (Fisher, 1922)\footnote{Ronald Aylmer Fisher (1890-1962), biólogo e estatístico inglês. Para a história do desenvolvimento do EMV, ver~\href{https://projecteuclid.org/euclid.ss/1030037906}{Aldrich (1997)}.}.
  Quando observamos $\boldsymbol{X} = \boldsymbol{x}$, dizemos que $\delta(\boldsymbol{x})$ é uma~\textit{estimativa} de $\theta$.
  Dito de outra forma,
   $$ \max_{\theta \in \Omega} f(X \mid \theta) = f(X\mid \hat{\theta}).$$
 \end{defn}
\end{frame}

\begin{frame}{Mudando de paradigma}
Na Definição~\ref{def:MLE}, vemos $\theta$ com um número real que indexa a distribuição de probabilidade conjunta dos dados.

\begin{itemize}
 \item Poderíamos trocar\footnote{Mas não vamos, pois a notação fica clara em quase todos os contextos.} $f(x \mid \theta)$ por $f(x; \theta)$;
 \item Com o EMV, procuramos um valor de $\theta$ de modo que a probabilidade de observarmos $\boldsymbol{X} = \boldsymbol{x}$ seja máxima;
 \item Isso não nos diz nada sobre o quão provável $\hat{\theta}$ é;
 \item $\theta$ não é uma quantidade aleatória, portanto não admite afirmações probabilísticas.
\end{itemize}
\end{frame}

\begin{frame}{Exemplos}
 \begin{itemize}
  \item Exponencial;
  \item Bernoulli;
  \item Normal;
  \begin{itemize}
   \item $\mu$ desconhecida, $\sigma^2$ conhecida;
   \item $\mu$ conhecida, $\sigma^2$ desconhecida;
   \item $\mu$ e $\sigma^2$ ambas desconhecidas.
  \end{itemize}
 \end{itemize}
\end{frame}

\begin{frame}{EMVs}
 \begin{itemize}
  \item Exponencial: $\hat{\theta} = 1/\bar{X}_n$;
  \item Bernoulli $\hat{\theta} = \bar{X}_n$;
  \item Normal;
  \begin{itemize}
   \item $\hat{\mu} = \bar{X}_n$;
   \item $\hat{\sigma^2} = \frac{1}{n}\sum_{i=1}^n (X_i - \mu)^2$;
   \item $\hat{\theta} = \left\{ \hat{\mu} = \bar{X}_n,  \hat{\sigma^2} = \frac{1}{n}\sum_{i=1}^n (X_i - \bar{X}_n)^2 \right\}$.
  \end{itemize}
 \end{itemize}
\end{frame}

\begin{frame}{EMV para uma distribuição uniforme}

\begin{exemplo}[EMV para uniforme]
\label{ex:uniform_closed}
 Suponha que $\rs$ perfazem uma amostra aleatória de uma distribuição uniforme no intervalo $[0, \theta]$, $\theta \in \mathbb{R}, \theta > 0$.
 Considere a f.d.p.
 \begin{equation}
  \label{eq:uniform_closed}
  f(x\mid \theta)=
 \begin{cases}
     \frac{1}{\theta}, 0 \leq x \leq \theta,\\
     0,\:\text{caso contrário}.
\end{cases}
  \end{equation}
A f.d.p. conjunta é
\begin{equation}
 \label{eq:uniform_closed_joint}
   f_n(\boldsymbol{x} \mid \theta)=
 \begin{cases}
     \theta^{-n}, 0 \leq x_i \leq \theta \: (i = 1, 2, \ldots, n),\\
     0,\:\text{caso contrário},
\end{cases}
\end{equation}
e o EMV é $\hat{\theta} = \max(x_1, x_2, \ldots, x_n)$.
\end{exemplo}
\end{frame}

\subsection*{Existência} 
\begin{frame}{Existência do EMV}
\begin{obs}[Existência do EMV]
\label{rmk:existence_MLE}
A existência do EMV pode depender de detalhes irrelevantes acerca do espaço de parâmetros, $\Omega$. 
\end{obs}
\begin{exemplo}[Não existência do EMV]
 Considere o Exemplo~\ref{ex:uniform_closed}, mas agora com uma f.d.p. um pouco diferente:
 \begin{equation}
  \label{eq:uniform_open}
    f(x\mid \theta)=
 \begin{cases}
     \frac{1}{\theta}, 0 < x < \theta,\\
     0,\:\text{caso contrário}.
\end{cases}
 \end{equation} 
É fácil mostrar que, nesse caso, o EMV não existe. 
\end{exemplo}
\end{frame}

\subsection*{Unicidade} 
\begin{frame}{Unicidade do EMV}
\begin{obs}[Unicidade do EMV]
\label{rmk:unicity_MLE}
Mesmo quando existe, o EMV nem sempre é único. 
\end{obs}

\begin{exemplo}[EMV para uma uniforme num intervalo de tamanho $1$]
 Suponha que $\rs$ perfazem amostra aleatória de uma distribuição uniforme no intervalo $[\theta, \theta + 1]$.
A densidade conjunta é
  \begin{equation}
  \label{eq:uniform_thetapOne}
    f_n(\boldsymbol{x}\mid \theta)=
 \begin{cases}
     1, \theta \leq x_i \leq \theta + 1, \: (i = 1, 2, \ldots, n),\\
     0,\:\text{caso contrário}.
\end{cases}
 \end{equation}
Defina $m := \min(x_1, x_2, \ldots, x_n)$ e $M := \max(x_1, x_2, \ldots, x_n)$.
Podemos reescrever~(\ref{eq:uniform_thetapOne}) como 
  \begin{equation}
  \label{eq:uniform_thetapOne_b}
    f_n(\boldsymbol{x}\mid \theta)=
 \begin{cases}
     1, M - 1 \leq \theta \leq m, \: (i = 1, 2, \ldots, n),\\
     0,\:\text{caso contrário}.
\end{cases}
 \end{equation}
\end{exemplo}
\textbf{Conclusão:} $\hat{\theta}$ é qualquer valor no intervalo $[M-1, m]$.
\end{frame}
 
\subsection*{Invariância} 
\begin{frame}{Invariância do EMV}
Suponha que estamos interessados em uma transformação do parâmetro $\theta$, $\phi(\theta)$.
Por exemplo, se $\rs$ são Bernoulli com parâmetro $p$,  podemos estar interessados na~\textit{chance} $\omega = \phi(p) = p/(1-p)$.
\begin{theo}[Invariância do EMV]
\label{thm:invariance_MLE}
 Considere uma função $\phi : \Omega \to \mathbb{R}$.
 Se $\hat{\theta}$ é um EMV para $\theta$, então $\phi(\hat{\theta})$ é um EMV para $\omega = \phi(\theta)$.
\end{theo}
\textbf{Prova:} Defina a~\textit{verossimilhança induzida}:
\[ L^\ast(\omega) := \sup_{\left\{ \theta: \phi(\theta) = \omega \right\}} L(\theta), \]
e note que o supremo desta função sobre $\Omega$ é precisamente o EMV.
Ver Casella \& Berger, Teorema 7.2.10 (pág. 320) ou DeGroot, Teorema 7.6.2 (pág 427).
\textbf{Exemplo:} O EMV para o quadrado da média de uma normal, $\mu^2$, é $\bar{X}_n^2$.
\end{frame}

\subsection*{Consistência} 

\begin{frame}{Consistência do EMV}
Sob condições de regularidade, o EMV é consistente, isto é $\hat{\theta}_{\text{EMV}} \xrightarrow{} \theta$.
\begin{theo}[Consistência do EMV]
\label{thm:consistency_MLE}
Defina $l(\theta) := \log f_n(\boldsymbol{x} \mid \theta)$ e assuma que $\rs \sim f(\theta_0)$, isto é, que $\theta_0$ é o valor verdadeiro do parâmetro.
Denote $E_{\theta_0}[g] := \int_{\mathcal{X}} g(x, \theta_0) f(\boldsymbol{x} \mid \theta_0)\, dx$.
 Suponha que 
 \begin{itemize}
  \item $f(x_i \mid \theta)$ tem o mesmo suporte;
  \item $\theta_0$ é ponto interior de $\Omega$;
  \item $l(\theta)$ é diferenciável;
  \item $\hat{\theta}_{\text{EMV}}$ é a única solução de $l^\prime(\theta) = 0$.
 \end{itemize}
   Então,
  $$\hat{\theta}_{\text{EMV}} \xrightarrow{} \theta.$$
\end{theo}
\textbf{Prova:} (rascunho) mostrar que, para todo $\theta \in \Omega$,
\[ \frac{1}{n} \sum_{i=1}^n \log f(X_i \mid \theta) \xrightarrow{} E_{\theta_0}\left[ \log f(\boldsymbol{X} \mid \theta) \right], \]
e aplicar a desigualdade de Jensen.
\end{frame}

\begin{frame}{O que aprendemos?}
\begin{itemize}
 \item[\faLightbulbO] Estimador de máxima verossimilhança (EMV);
 
   ``Encontrar o valor do parâmetro que maxima a probabilidade observar os dados obtidos''
   
  \item[\faLightbulbO] Invariância ;
  
  ``O EMV é invariante a transformações dos parâmetros; se $\hat{\theta}$ é o EMV para $\theta$, $\phi(\hat{\theta})$ é o EMV para $\phi(\theta)$''
  
    \item[\faLightbulbO] Consistência;
  
  ``Sob condições brandas de regularidade, o EMV converge para valor verdadeiro à medida que $n \to \infty$''
  
    \item[\faLightbulbO] Limitações;
  
  ``O EMV não existe necessariamente, e, mesmo quando existe, não precisa ser único''
  
  \end{itemize}
 \end{frame}

\begin{frame}{Leitura recomendada}
\begin{itemize}
 \item[\faBook] DeGroot seções 7.5 e 7.6;
 \item[\faBook] $^\ast$ Casella \& Berger, seção 7.2.2.
 \item[\faBook] $^\ast$ Schervish (1995), seção 5.1.3.
   \item[\faForward] Próxima aula: DeGroot, seção 7.6 (pág. 432 em diante);
 \item {\large\textbf{Exercícios recomendados}}
 \begin{itemize}
  \item[\faBookmark] DeGroot,
  \begin{itemize}
   \item Seção 7.5: exercícios  1, 4, 9 e 10;
   \item Seção 7.6: exercícios 3, 5 e 11.
  \end{itemize}   
  \end{itemize}
 \end{itemize} 
\end{frame}

