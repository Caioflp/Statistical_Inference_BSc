\section*{Suficiência}
\begin{frame}{Suficiência}
 \begin{itemize}
  \item Estatística suficiente;
  \item Teorema da fatorização;
    \item Suficiência conjunta;
  \item Suficiência mínima;
 \end{itemize}
\end{frame}

\begin{frame}{Um exemplo motivador}
Suponha que os tempos de falha de um modelo de lâmpada podem ser modelados como $\rs \sim \operatorname{expo}(\theta)$.

Suponha que dois técnicos, Afonso e Bruna, medem cada um três lâmpadas, obtendo:
\begin{itemize}
 \item $\boldsymbol{x}_{\text{A}} = \{1.64, 1.37, 0.13\}$ meses;
 \item $\boldsymbol{x}_{\text{B}} = \{0.48, 0.87, 1.79\}$ meses;
\end{itemize}

O chefe dos dois, Astolfo, suspeita que o tempo de falha seja, em média, 2 meses com desvio padrão de mais ou menos 1 mês.
Para cada uma das amostras
\begin{itemize}
 \item (i) Compute o estimador de Bayes $\theta$ sob perda quadrática; 
 \item (ii) Estime $\theta$ por máxima verossimilhança.
\end{itemize}
\end{frame}

\begin{frame}{Estatística suficiente}
\begin{defn}[Estatística suficiente]
 \label{def:sufficient_statistic}
 Seja $\rs$ uma amostra aleatória de uma distribuição indexada pelo parâmetro $\theta$.
 Seja $T = r(\rs)$ uma estatística.
 Dizemos que $T$ é uma~\textbf{estatística suficiente} para $\theta$ se e somente se
 \[ f(\rs \mid T = t, \theta) = f(\rs \mid T = t, \theta^\prime),\: \forall\, \theta, \theta^\prime \in \Omega, \]
 isto é, se a distribuição condicional da amostra dado o valor da estatística não depende de $\theta$.
\end{defn}

No exemplo anterior, tanto $\hat{\theta}_{\text{Bayes}}$ quanto $\hat{\theta}_{\text{EMV}}$ dependem de $\rs$ apenas através de $r(\rs) = T = \sum_{i=1}^n X_i$.
\end{frame}

\begin{frame}{Uma observação importante}

\begin{defn}[Aleatorização auxiliar]
 \label{def:auxiliary_randomisation}
 Suponha que $T$ é suficiente para $\theta$. 
 O processo de simular $X_1^\prime, \ldots, X_n^\prime \mid T = r(\rs)$ de modo que
 \[ f(\rs \mid \theta) = f(X_1^\prime, \ldots, X_n^\prime \mid \theta),\, \forall \theta \in \Omega, \]
é chamado de~\textbf{aleatorização auxiliar} (em inglês,~\textit{auxiliary randomisation}).
\end{defn}
\begin{obs}[A busca por bons estimadores]
\label{rmk:good_estimators_sufficient}
 Na busca por bons estimadores, estamos justificados em restringir a busca a funções de estatísticas suficientes.
\end{obs}
\textbf{Justificativa:} Suponha que o estatístico A tem à sua disposição $\rs$, enquanto B tem acesso somente a $T  = r(\rs)$.
Se $T$ é suficiente, B pode sempre fazer uma aleatorização auxiliar e gerar $X_1^\prime, \ldots, X_n^\prime$ com exatamente a mesma distribuição conjunta condicional a $\theta$.
\end{frame}

\begin{frame}{Teorema da fatorização (TF)}
\begin{theo}[Teorema da fatorização]
 \label{thm:factorisation}
Suponha que $\rs$ perfazem uma amostra aleatória com f.d.p./f.m.p $f(x \mid \theta)$, $\theta \in \Omega$.
Uma estatística $T = r(\rs)$ é suficiente para $\theta$ se, e somente se, para todo $\boldsymbol{x} \in \mathcal{X}$ e $\theta \in \Omega$ existem $u$ e $v$ não negativas tal que
\begin{equation*}
 f_n(\boldsymbol{x} \mid \theta) = u(\boldsymbol{x}) v[r(\boldsymbol{x}), \theta].
\end{equation*}
\end{theo}
\textbf{Prova:} (Para v.a.s discretas).
Para a ``ida'' notar que $T$ é uma função determinística de $X$, ou seja, $\pr(T = t \mid \boldsymbol{X} = \boldsymbol{x} , \theta) = 1$ e que só precisamos considerar $\boldsymbol{x} \in \left\{ \boldsymbol{y} : r(\boldsymbol{y}) = t \right\}$.
Para a ``volta'', mostrar que $T$ suficiente implica que $\pr(\boldsymbol{X} = \boldsymbol{x} \mid T = t, \theta)$ é função apenas de $\boldsymbol{x}$.
Ver DeGroot, Teorema 7.7.1 e Casella \& Berger, Teorema 6.2.6.
\end{frame}

\begin{frame}{O TF em ação}
\begin{itemize}
 \item Poisson;
 \item $f(x\mid\theta) = \theta x^{\theta-1}$, $x \in (0, 1)$ e $\theta > 0$;
 \item Normal;
\end{itemize} 
\end{frame}

\begin{frame}{Suficiência conjunta}
O que acontece, por exemplo, no caso Normal com $\mu$ e $\sigma^2$ desconhecidos?

 \begin{defn}[Suficiência conjunta]
  \label{def:jointly_sufficient}
  Dizemos que  um conjunto de estatísticas $\boldsymbol{T} = \{T_1, \ldots, T_k \}$ é~\textbf{suficiente} (conjuntamente) se que a distribuição condicional conjunta de $\rs$ dado $T_1 = t_1, \ldots, T_k = t_k$ não depende de $\theta$.  
 \end{defn}
 \begin{obs}[TF para estatísticas suficientes conjuntas]
  Para o caso de estatísticas suficientes conjuntas, vale um Teorema da fatorização:
  \begin{equation*}
 f_n(\boldsymbol{x} \mid \theta) = u(\boldsymbol{x}) v[r_1(\boldsymbol{x}), \ldots, r_k(\boldsymbol{x}), \theta].
\end{equation*}
 \end{obs}

\end{frame}

\begin{frame}{Suficiência conjunta -- exemplos}
 \begin{itemize}
  \item Normal;
  \item Uniforme;
 \end{itemize}

  \begin{obs}[Transformações biunívocas de estatísticas suficientes]
    Se $\boldsymbol{T} =  \{T_1, \ldots, T_k \}$ são estatísticas suficientes conjuntas, e $h : \mathcal{T} \to \mathbb{R}$ é um mapa inversível, então $\boldsymbol{T^\prime} = h(\boldsymbol{T})$ também são suficientes conjuntas. 
 \end{obs}
\end{frame}

\begin{frame}{Suficiência mínima -- motivação}
Primeiro um exemplo motivador:
\begin{defn}[Estatísticas de ordem]
 \label{def:order_statistics}
Seja $\boldsymbol{X} = \rs$ uma amostra aleatória.
Dizemos que $Y_1, Y_2, \ldots, Y_n$ são~\textbf{estatísticas de ordem} se $Y_1$ é o menor valor de $\boldsymbol{X}$, $Y_5$ é o quinto menor valor e assim por diante.
\end{defn}
\begin{theo}[Estatísticas de ordem são suficientes conjuntas]
Seja $\rs$ uma amostra aleatória com f.d.p/f.m.p. $f(x\mid\theta)$.
As estatísticas de ordem $Y_1, Y_2, \ldots, Y_n$ são suficientes conjuntas para $\theta$.
\end{theo}
\textbf{Prova:} Usar o fato de que a conjunta é o produto das marginais e a comutatividade da multiplicação em $\mathbb{R}$. 
\end{frame}

\begin{frame}{Suficiência mínima}
\begin{defn}[Suficiência mínima]
 \label{def:minimal_sufficiency}
 Uma estatística $T$ é dita~\textbf{mínima suficiente} se $T$ é suficiente e é função de qualquer outra estatística suficiente.
 Um vetor $\boldsymbol{T} =  \{T_1, \ldots, T_k \}$ é dito~\textbf{minimamente suficiente conjunto} se é função de qualquer outro vetor de estatísticas suficientes conjuntas.
\end{defn}
\begin{obs}[Estatísticas de ordem são minimamente suficiente conjuntas no caso Cauchy]
 \begin{equation}
  f_n(\boldsymbol{x} \mid \theta) = \frac{1}{\pi^n \prod_{i=1}^n\left[1 + (x_i-\theta)^2\right]}
 \end{equation}
\end{obs}
\end{frame}

\begin{frame}{EMV e Bayes como estatísticas minimamente suficientes}
 \begin{theo}[EMV e Bayes são suficientes]
 \label{thm:MLE_Bayes_sufficient}
  Se a função de verossimilhança admite fatorização como no Teorema~\ref{thm:factorisation}, os  estimadores de Bayes e de máxima verossimilhança são estatísticas minimamente suficientes.
 \end{theo}
\textbf{Prova:}
\begin{itemize}
 \item EMV: notar que $f_x(\boldsymbol{x} \mid \theta) \propto v[r(\boldsymbol{x}), \theta]$;
 \item Bayes: escrever a perda esperada~\text{a posteriori} explicitamente usando a verossimilhança na forma do TF.
\end{itemize}
Ver Teoremas 7.8.3 e 7.8.4 de DeGroot.
\end{frame}


\begin{frame}{O que aprendemos?}
\begin{itemize}
  \item[\faLightbulbO] Estatística suficiente;
    
    ``Uma estatística $T$ é suficiente para $\theta$ se $\pr(\boldsymbol{X} = \boldsymbol{x} \mid T = t, \theta)$ não depende de $\theta$.''
    
   \item[\faLightbulbO] Teorema da fatorização;
   
   ``Se $T$ é suficiente para $\theta$, podemos escrever a verossimilhança como o produto entre uma função que não depende de $\theta$ e uma função que só depende de $\boldsymbol{X}$ através de $T$.'' 
   
     \item[\faLightbulbO] Os estimadores de Bayes e de máxima verossimilhança são minimamente suficientes.   
    
  \end{itemize}
 \end{frame}

\begin{frame}{Leitura recomendada}
\begin{itemize}
 \item[\faBook] DeGroot seções 7.7 e 7.8;
 \item[\faBook] $^\ast$ Casella \& Berger (2002), seção 6.2.
%  \item[\faBook] $^\ast$ Schervish (1995), capítulo 7.
 \item[\faForward] Próxima aula: DeGroot, seção 7.9;
 \item {\large\textbf{Exercícios recomendados}}
 \begin{itemize}
  \item[\faBookmark] DeGroot.
  \begin{itemize}
   \item Seção 7.7: exercícios 4, 7, 13, 16;
   \item Seção 7.8: exercícios 3, 8, 12, 16.
  \end{itemize}   
  \end{itemize}
 \end{itemize} 
\end{frame}
