\section{Testes para igualdade de variâncias}
\begin{frame}{Testes para igualdade de variâncias}
 \begin{itemize}
   \item A distribuição F;
   \item Comparação de variâncias de duas normais;
   \item Propriedades; 
   \item P-valor;
   \end{itemize}
\end{frame}


 \begin{frame}{A distribuição F}
  Sejam $Y \sim\operatorname{Qui-quadrado}(m)$ e $W \sim\operatorname{Qui-quadrado}(n)$.
  Então 
  \begin{equation*}
   X = \frac{Y/m}{W/n},
  \end{equation*}
tem distribuição $F$ com $m$ e $n$ graus de liberdade, com f.d.p.
\begin{equation*}
    f_X(x) = \frac{\Gamma\left(\frac{m + n}{2}\right)m^{m/2} n^{n/2}}{\Gamma\left(\frac{n}{2}\right)\Gamma\left(\frac{m}{2}\right)} \cdot \frac{x^{m/2-1}}{(mx + n)^{(m + n)/2}}, \: x > 0.
\end{equation*}
\begin{theo}[Propriedades da distribuição F]
\label{thm:F_distribution_properties}
 \begin{itemize}
  \item[i)] Se $X \sim F(m, n)$, então $\frac{1}{X} \sim F(n, m)$;
  \item[ii)] Se $Y \sim\operatorname{Student}(n)$, então $Y^2 \sim F(1, n)$.
 \end{itemize}
\end{theo}
\textbf{Prova:} Transformação de v.a.s padrão.
Exercício para a leitora.
 \end{frame}

 \begin{frame}{Testando a igualdade de duas variâncias}
  Suponha $X_i \sim\operatorname{Normal}(\mu_1, \sigma_1^2), i = 1, 2, \ldots, m$ e $Y_j \sim\operatorname{Normal}(\mu_2, \sigma_2^2), j = 1, 2, \ldots, n$.
  Estamos interessados em testar
  \begin{align*}
   H_0 &: \sigma_1^2 \leq \sigma_2^2 , \\
   H_1&:  \sigma_1^2 > \sigma_2^2. 
  \end{align*}
 Para isso, vamos computar a estatística de teste 
 \begin{equation*}
  V = \frac{S_X^2/(m-1)}{S_Y^2/(n-1)},
 \end{equation*}
 onde $S_X^2 = \sum_{i=1}^m (X_i-\bar{X}_m)^2$ e $S_Y^2 = \sum_{j=1}^n (Y_j-\bar{Y}_n)^2$.
 
 \begin{defn}[O teste F]
 \label{def:F_test}
  O teste F de homogeneidade (igualdade de variâncias)  é o teste $\delta_c$ que rejeita $H_0$ se $V \geq c$, para uma constante positiva $c$.
 \end{defn}

 \end{frame}

 \begin{frame}{Propriedades do teste F}
  Em primeiro lugar, podemos fazer afirmações sobre a distribuição de (uma transformação de) $V$.
  \begin{theo}[A distribuição de $V$]
  \label{thm:V_distribution}
   Seja $V = \frac{S_X^2/(m-1)}{S_Y^2/(n-1)}$, então:
   \begin{equation*}
    \frac{\sigma_2^2}{\sigma_1^2} V \sim F(m-1, n-1).
   \end{equation*}
   Além disso, se $\sigma_1^2 = \sigma_2^2$, $V \sim F(m-1, n-1)$. 
  \end{theo}
\textbf{Prova:} Notar que $S_X^2/\sigma_1^2$ e $S_Y^2/\sigma_2^2$  tem distribuição qui-quadrado com $m-1$ e $n-1$ graus de liberdade, respectivamente.
Ver Teorema 9.7.3 de DeGroot.
 \end{frame}
 
 \begin{frame}{P-valor}
  Seja $G(x; m-1, n-1)$ a f.d.a. de uma distribuição $F$ com $m-1$ e $n-1$ graus de liberdade.
  Da mesma forma, defina $G^{-1}(p; m-1, n-1)$ como a f.d.a. inversa.
  Então, se $V = v$:
  \begin{itemize}
   \item Para a hipótese $H_0: \sigma_1^2 \leq \sigma_2^2$, o p-valor vale $p = 1-G(v; m-1, n-1)$;
   \item Para a hipótese $H_0: \sigma_1^2 \geq \sigma_2^2$, o p-valor vale $p = G(v; m-1, n-1)$;
   \item Para a hipótese bicaudal $H_0: \sigma_1^2 \neq \sigma_2^2$, o p-valor vale $p = 2\min\left\{1-G(v; m-1, n-1), G(v; m-1, n-1)\right\}$;
  \end{itemize}
 \end{frame}
 
  \begin{frame}{Mais propriedades do teste F}
 Analogamente ao teste t, podemos enunciar o seguinte teorema sobre o teste F.
 \begin{theo}[Propriedades do teste F]
 \label{thm:F_test_properties}
 Suponha que estamos testando $H_0: \sigma_1^2 \leq \sigma_2^2$.
 Então
 \begin{itemize}
 \item [i)] $\pi(\mu_1, \mu_2, \sigma_1^2, \sigma_2^2 \mid \delta_c) = 1 -G\left(\frac{\sigma_2^2}{\sigma_1^2}c; m-1, n-1\right)$;
 \item [ii)] $\sigma_1^2 = \sigma_2^2 \implies \pi(\mu_1, \mu_2, \sigma_1^2, \sigma_2^2 \mid \delta_c) = \alpha_0$;
 \item [iii)] $\sigma_1^2 < \sigma_2^2 \implies \pi(\mu_1, \mu_2, \sigma_1^2, \sigma_2^2 \mid \delta_c) < \alpha_0$
 \item [iv)] $\sigma_1^2 > \sigma_2^2 \implies \pi(\mu_1, \mu_2, \sigma_1^2, \sigma_2^2 \mid \delta_c) > \alpha_0$;
 \item [v)] $\lim_{\sigma_1^2/\sigma_2^2 \to 0} \pi(\mu_1, \mu_2, \sigma_1^2, \sigma_2^2 \mid \delta_c) = 0$;
 \item [vi)] $\lim_{\sigma_1^2/\sigma_2^2 \to \infty} \pi(\mu_1, \mu_2, \sigma_1^2, \sigma_2 \mid \delta_c) = 1$;
 \item[vii)] $\delta_c$ é não-viesado e tem tamanho $\alpha_0$.
\end{itemize}
\end{theo}
\textbf{Prova:} Omitida aqui. 
Ver Teorema 9.7.4 de DeGroot.  
 \end{frame}
 
\begin{frame}{O que aprendemos?}
\begin{itemize}
  \item[\faLightbulbO] A distribuição F aparece quando tomamos a razão de variáveis aleatórias Qui-quadrado;    
  \item[\faLightbulbO] Para comparação das variâncias de duas amostras a estatística teste tem distribuição $F$ com $m-1$ e $n-1$ graus de liberdade; 
  \item O teste F, como seu primo o teste t, é não viesado e tem tamanho $\alpha_0$.
   \end{itemize}
 \end{frame} 
 
\begin{frame}{Leitura recomendada}
\begin{itemize}
 \item[\faBook] DeGroot seção 9.7;
 \item[\faBook] $^\ast$ Casella \& Berger (2002), seção 8.
 \item[\faForward] Próxima aula: DeGroot, seção 11;
 \item {\large\textbf{Exercícios recomendados}}
  \begin{itemize}
   \item[\faBookmark] Derivar a função de densidade de probabilidade de uma distribuição F (Teorema 9.7.1 de DeGroot).
   \item[\faBookmark] Derivar o teste F como um teste de razão de verossimilhanças.
  \end{itemize}
 \end{itemize} 
\end{frame}
