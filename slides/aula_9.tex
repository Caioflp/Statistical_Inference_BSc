\section*{Erro quadrático médio e Rao-Blackwell}
\begin{frame}{EQM e Rao-Blackwell}

Como avaliar um estimador?

\begin{defn}[Notação conveniente]
Para as próximas computações, é conveniente definir
Para $g : \mathcal{X}^n \to \mathbb{R}$, escrevemos
\[ E_{\theta} [g] = \int_{\mathcal{X}} \cdots \int_{\mathcal{X}} g(\bx)f_n(\bx \mid \theta)\, dx_1\cdots\,dx_n = \int_{\mathcal{X}} g(\bx)f_n(\bx \mid \theta) \,d\bx. \] 
\end{defn}

Agora podemos definir o~\textbf{erro quadrático médio} (EQM) de um estimador $\delta(\bX)$:
\begin{defn}[Erro quadrático médio]
 \label{def:MSE}
 \begin{equation*}
  R(\theta, \delta) := E_{\theta} \left[\left\{\delta(\bX) - \theta\right\}^2\right].
 \end{equation*}
\end{defn} 
\end{frame}

\begin{frame}{Condicionando em uma estatística suficiente}
 Seja $\bT$ uma estatística suficiente.
 Podemos definir o seguinte estimador
 \begin{defn}[Estimador condicionado]
 \begin{equation*}
 \label{def:conditioned_estimator}
  \delta_0(\bT) := E_{\theta} \left[ \delta(\bX) \mid \bT \right].
 \end{equation*}  
 \end{defn}
 Como $\bT$ é suficiente, podemos escrever, simplesmente,
  \begin{equation*}
  \delta_0(\bT) = E \left[ \delta(\bX) \mid \bT \right].
 \end{equation*}  
\end{frame}

\begin{frame}{O Teorema de Rao-Blackwell}
 Com essas definições em mãos, estamos preparados para enunciar um dos teoremas mais importantes da Estatística:
 \begin{theo}[Teorema de Rao-Blackwell\footnote{O estatístico indo-estadunidense Calyampudi Radhakrishna Rao (1920-) e o estatístico estadunidense David Harold Blackwell (1919-2010) provaram o resultado independentemente no final dos anos 1940.}]
  \label{thm:Rao-Blackwell}
  Seja $\delta(\bX)$ um estimador, $\bT$ uma estatística suficiente para $\theta$ e seja $\delta_0(\bT)$ como na definição~\ref{def:conditioned_estimator}. 
  Então vale que
  \begin{equation*}
   R(\theta, \delta_0) \leq R(\theta, \delta).
  \end{equation*}
 \end{theo}
\end{frame}

\begin{frame}{Prova do TRB}
 Primeiro, notemos que, para qualquer função $g$ e variáveis aleatórias $X$ e $Y$, valem os seguintes fatos:
 \begin{itemize}
  \item $\left(E[g(X) \mid Y] \right)^2 \leq E\left[\{g(X)\}^2 \mid Y\right]$;
  
  Desigualdade de Cauchy-Schwarz\footnote{Em homenagem ao matemático francês Augustin-Louis Cauchy (1789-1857) e ao matemático alemão Karl Hermann Amandus Schwarz (1843-1921).}, também obtida, nesse caso, rearranjando a expressão da variância.
  
  \item $E\left\{E[X\mid Y]\right\} = E[X]$ (lei da esperança total).
 \end{itemize}

 Fazendo $g(X) = \left( \delta(\bX)-\theta\right)^2$, obtemos
 \begin{equation}
 \label{eq:RB_ineq1}
  \left(E\left[\delta(\bX) \mid \bT \right] -\theta \right)^2 \leq E\left[\left( \delta(\bX) - \theta \right)^2 \mid \bT \right]
 \end{equation}
Note que $\left(E\left[\delta(\bX) \mid \bT \right] -\theta \right)^2 = \left[\delta_0(\bT) -\theta \right]^2$.
Agora, tomamos esperanças nos dois lados de~(\ref{eq:RB_ineq1}) para obter:
\begin{align*}
  R(\theta, \delta_0) &= E\left[ \left(\delta_0(\bT) -\theta \right)^2 \right] \leq E\left\{E\left[\left\{ \delta(\bX) - \theta \right\}^2 \mid \bT \right]\right\} \\
  &= E\left[ \left\{\delta(\bX) -\theta \right\}^2 \right] = R(\theta, \delta). \qed
\end{align*}
\end{frame}

\begin{frame}{Admissibilidade}
O conceito de admissibilidade diz respeito à relação entre estimadores.
\begin{defn}[Admissibilidade]
 \label{def:admissibility}
 Um estimador $\delta$ é dito~\textbf{inadmissível} se existe outro estimador $\delta_0$ tal que $R(\theta, \delta_0) \leq R(\theta, \delta)$ para todo $\theta \in \Omega$ e existe $\theta^\prime \in \Omega$ tal que $R(\theta^\prime, \delta_0) < R(\theta^\prime, \delta)$.
 Nesse caso, dizemos que $\delta_0$~\textit{domina} $\delta$.
 O estimador $\delta_0$ é~\textbf{admissível} se (e somente se) não há nenhum estimador que o domine.
\end{defn}

\begin{obs}[Estimadores admissíveis e o Teorema de Rao-Blackwell]
 O Teorema de Rao-Blackwell diz que todo estimador condicionado em uma estatística suficiente é admissível.
\end{obs}

\begin{exemplo}[Estimadores no caso normal]
\begin{itemize}
 \item Estimando $\mu$ através da mediana amostral;
 \item Estimando $\sqrt{\sigma^2}$.
\end{itemize}
\end{exemplo}

\end{frame}


\begin{frame}{O que aprendemos?}
\begin{itemize}

  \item[\faLightbulbO] Teorema de Rao-Blackwell;    
  
    ``Quando $\bT$ é uma estatística suficiente, todo estimador condicionado em $\bT$ tem menor EQM''
    
 \item[\faLightbulbO] Estimador admissível;
 
  ``Um estimador é admissível quando domina todos os outros estimadores ''

 \item[\faLightbulbO] Caso normal;
 
  ``No caso normal, qualquer estimador de $\mu$ que não seja função de $\bar{X}_n$ é inadmissível.
  O mesmo vale para qualquer estimador de $\sqrt{\sigma^2}$ que não seja função de $\sum_{i=1}^n X_i$ e $\sum_{i=1}^n X_i^2$.''
 
 
  \end{itemize}
 \end{frame}

\begin{frame}{Leitura recomendada}
\begin{itemize}
 \item[\faBook] De Groot, seção 7.9;
 \item[\faBook] $^\ast$ Casella \& Berger (2002), seção 7.3.
 \item[\faBook] $^\ast$ Schervish (1995),  Teorema 3.20.
 \item {\large\textbf{Exercícios recomendados}}
 \begin{itemize}
  \item[\faBookmark] De Groot, Seção 7.9: exercícios 2, 3, 6 e 10.
  \end{itemize}
 \end{itemize} 
\end{frame}
